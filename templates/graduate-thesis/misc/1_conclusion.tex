%%
% The BIThesis Template for Graduate Thesis
%
% Copyright 2020-2023 Yang Yating, BITNP
%
% This work may be distributed and/or modified under the
% conditions of the LaTeX Project Public License, either version 1.3
% of this license or (at your option) any later version.
% The latest version of this license is in
%   http://www.latex-project.org/lppl.txt
% and version 1.3 or later is part of all distributions of LaTeX
% version 2005/12/01 or later.
%
% This work has the LPPL maintenance status `maintained'.
%
% The Current Maintainer of this work is Feng Kaiyu.

\begin{conclusion}
    本文面向工业实际中非平整表面的缺陷检测问题,基于结构光相机和无监督深度学习相关技术,研究了未知缺陷条件下基于3D点云数据场景和多模态数据场景的缺陷检测算法,并探索利用联邦学习实现算法的分布式部署。本文所用研究成果均在实验平台上进行了验证分析,其结果表明本文提出的算法可以提高面向非平整表面缺陷检测算法的性能。论文的主要工作及创新点梳理如下:

    (1)开发了自动数据采集系统。传统数据采集通过固定机位采集数据,然后通过数据增强来模拟真实工况下被检物的位移和旋转。本文创新地利用可编程协作机器人挂载结构光相机,在机械臂和结构光相机SDK的基础上开发了可以自动从多角度采集数据,并按标准格式存储为数据集的具有人机交互界面的数据采集系统。通过该系统可以更高效的采集高质量的数据集且生成的数据集相较于软件数据增强方法更接近真实工况。
    
    (2)提出了基于3D点云数据的缺陷检测算法,称为PillarCore。本文通过创新地引入RoPS点云局部特征描述子来对点云数据进行特征提取,从而将应用于二维图像未知缺陷检测的算法迁移到点云数据场景下,实现面向非平整表面复杂结构工件且仅有正常样本条件下的缺陷检测算法。PillarCore在本文的数据集和公开数据集上均进行了验证,实验结果相较于对比方法,在部分类别上本文缺陷检测性能提升明显。
    
    (3)提出了基于RGB-D的多模态缺陷检测算法,称为HF-AST。本文创新地引入HOG描述子对深度图像提取特征,并通过改进原AST算法中学生网络模型的学生网络,即加入针对深度特征进行卷积再与预训练网络提取的RGB特征图融合。改进后的HF-AST网络在本文的塑料件数据集上缺陷检测性能超过了改进前,且本文的消融实验证明本文引入HOG和加入深度特征卷积两个改进均对性能提升有作用。
    
    (4)应用联邦学习实现分布式缺陷检测。本文将联邦学习与异构教师学生网络结合,通过在客户端节点使用本地教师网络,而在服务器端全局聚合优化学生网络并与所有客户端节点共享来实现低通讯负载的分布式缺陷检测。本文在联邦学习仿真框架上模拟多个产品线部署了本文分布式缺陷检测算法,实验结果表明该方法能够实现分布式带来的可扩展性。
    
    本文在仅有正常样本且缺陷未知的条件下,提出了面向非平整表面实际场景的缺陷检测算法并探索了分布式部署。不过工业实际场景是多种多样的,本文的研究只能应对部分场景,要将本文成果应用到更多场景下仍需更多后续工作,现就本文存在的不足以及未来可以扩展的工作方向总结如下:
    
    (1)复杂背景下的前景提取。在背景较为简单或固定的工件上采集数据,并经过本文的前景提取处理后送入缺陷检测算法,其检测结果已达到了一定的性能。但在面对复杂背景时,本文目前的前景提取方法较为简单,无法准确地将被检物从复杂背景中提取出来,因此引入更加高性能的分割网络作为本文缺陷检测的上游任务是后续扩展本文所提方法在更多工业场景下应用的后续工作方向。
    
    (2)多个目标实现分类检测。本文目前仅研究了对单次单个目标进行缺陷检测,适用于相对较大的工件场景。由于本文没有多目标检测和分割的前置任务处理模块,因此对于工业实际生产中单次多个目标的缺陷检测场景,本文的缺陷检测算法目前不能处理。本文的后续工作可以将多目标检测分割网络与本文的缺陷检测算法相结合。
    
    (3)数据集不分类的完全无监督。完全无监督是指训练集混杂了正常与缺陷样本,却没有任何标注,而本文目前的无监督方法要求人为设计训练集,保证其只包含正常样本,相当于对正常样本进行了标注。因此通过使用完全无监督的设置可以进一步节约标注成本,并适用于样本极不均衡的工业数据,是本文后续可以探索扩展的一个研究方向。
\end{conclusion}
