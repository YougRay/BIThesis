%%
% The BIThesis Template for Graduate Thesis
%
% Copyright 2020-2023 Yang Yating, BITNP
%
% This work may be distributed and/or modified under the
% conditions of the LaTeX Project Public License, either version 1.3
% of this license or (at your option) any later version.
% The latest version of this license is in
%   http://www.latex-project.org/lppl.txt
% and version 1.3 or later is part of all distributions of LaTeX
% version 2005/12/01 or later.
%
% This work has the LPPL maintenance status `maintained'.
%
% The Current Maintainer of this work is Feng Kaiyu.

\chapter{绪论}

\section{本论文研究的目的和意义}
工业制品是现代社会中不可或缺的一部分,随着人们日益增长的美好生活需求,对工业制品品质的要求也越来越高,工业缺陷检测作为保障产品质量、维持生产稳定的重要技术之一,得到了越来越多的关注和重视。工业缺陷检测是指发现各种工业制品的外观可见缺陷,这些缺陷虽然微小,但可能严重危害产品的正常功能。工业制品的缺陷主要发生在产品的生产阶段,在以往的缺陷检测中,需要人工筛查,成本高、效率低,难以满足大规模质检需求。得益于工业成像、计算机视觉和深度学习等领域的不断创新,基于视觉的工业缺陷检测技术也得到了长足的发展,广泛应用于各个领域,为现代工业生产提供了更加高效、精准的质量保障手段。

基于视觉的工业缺陷检测技术高速发展的同时也面临着新的挑战和问题。对于诸如织物一类的平整表面,其表面几何结构通常不复杂,缺陷类别以纹理为主,针对这类问题的缺陷检测算法相对成熟且应用效果不错。但在工业实际问题中,需要质量检测的产品很少是平整的表面,大部分是曲率复杂、表面非平整的复杂几何结构,而针对此问题的相关研究较少,且应用于平整表面的缺陷算法也难以适用。因此,作为智能制造领域重要的基础研究与技术之一,面向非平整表面的工业缺陷检测技术不仅具有非常重要的研究价值,同时也拥有广阔的应用前景。

目前大部分缺陷检测的方法仅关注于单一的 RGB、灰度图像或其他成像形式的输入,这些输入在纹理类别上表现优秀,但其在几何结构类上任然有很大提升空间。因此在工业上,利用不同成像条件的差异产生区分力是一个正在探索的方向。比如在工业实际场景下,通过结构光、激光等方式获得的 3D 数据能进一步提升产品结构性缺陷的表达能力。

工业缺陷检测的另一个挑战是实际工况中常常面临只有正常样本,不易获取含有缺陷的样本。这使得产品的缺陷具有未知性与无规则性, 基于缺陷先验知识的方法存在较大的局限性。

因此本文针对工业实际痛点和需求,开展面向缺陷未知非平整表面的缺陷检测问题的研究实践。本文整体采用无监督的方法(即仅有正常样本),聚焦于利用几何结构表达能力更强的3D数据来进行缺陷检测,旨在解决缺陷未知条件下,面向非平整表面缺陷检测中复杂几何结构带来的问题,提升工业缺陷检测的性能。此外,为了综合不同成像条件的感知能力来应对更复杂的场景, 本文还通过融合多模态数据的特征来增强缺陷检测性能。

\section{国内外研究现状及发展趋势}
缺陷检测作为保证工业产品质量的重要环节,在工业领域具有广泛的应用。近年来,随着计算机视觉技术和深度学习的发展,基于视觉的非接触缺陷检测研究得到了国内外学者的广泛关注。本小节将从未知缺陷检测算法、特征提取算法和联邦学习算法三个方面出发介绍国内外研究现状及发展趋势。


\subsection{基于深度学习的未知缺陷检测研究}
目前基于计算机视觉,包括二维和三维图像,深度学习的缺陷检测方法主要是采用无监督的方式,即输入的训练数据仅为正常样本,模型在未标注的数据上训练,获取正常样本的内在特征和联系,缺陷被定义为正常范围之外的模式。无监督方法可以分为基于图像相似度的方法与基于特征相似度的方法。\cite{bergmannMVTec3DADDataset2022,chenSurfaceDefectDetection2021,zhengBenchmarkingUnsupervisedAnomaly2022,TaoXianJiYuShenDuXueXiDeBiaoMianQueXianJianCeFangFaZongShu2021a,luoGongyequexianjianceshenduxuexifangfazongshu2022}

\subsubsection{基于图像相似度}

基于图像相似度的方法在图像像素层面进行比较。其核心思想是重建出与输入样本最相近的正常图像,两者仅在缺陷区域存在差别。因此,生成图与输入图像的差异图可表示缺陷存在的概率,既可以用于判断整图是否包含异常,也可以设定阈值来得到缺陷的分割结果。此类方法常使用自编码模型与生成式模型,包括自编码器 (auto-encoder, AE)\cite{hintonAutoencodersMinimumDescription1993}、变分自编码器 (variational auto-encoder, VAE)\cite{kingmaAutoEncodingVariationalBayes2022}、生成对抗模型 (generative adversarial networks, GAN)\cite{hoGenerativeAdversarialImitation2016} 等。根据模型优化目标的不同,可分为基于图像重建的方法与基于图像恢复的方法。

基于图像重建的方法假设只在正常样本上训练模型,并使其学习足够好的分布特征来重建正常样本,那么模型应该只能较好地重建正常样本,而在缺陷区域则会产生大量的重建误差。该方法的缺陷检测性能受重建图像的质量影响,通过提高重建质量使重建图像与输入图像在缺陷区域差异凸显是目前主流改进方向。

基于 AE 的方法采用编码器-解码器结构的网络,编码器将输入图像编码为隐空间变量 z,解码器则利用 z 重建图像。通过计算输入图像与重建图像的重建误差,可以实现缺陷定位。然而,AE 在重建时存在模糊的现象,容易在计算重建误差时误检成正常像素点。提升重建图像的质量和利用模糊效应,将输入图像风格化是后续工作改进的方向。

基于VAE的方法利用编码器将正常图像映射到隐空间的先验分布,从对应分布中随机采样得到变量z,再利用解码器将z映射回图像空间。该方法假设缺陷区域的编码不服从该先验分布,无法较好地重建缺陷区域,从而可根据重建误差实现缺陷定位。由于可以构建比AE更加可控的结构化的隐空间流形,VAE可以从隐空间分布的角度来学习正常样本主要的共性特征。

基于 GAN 的方法利用了其对分布建模的强大能力,相比 AE 和 VAE,GAN 能够生成更清晰、高质量的图像。GAN 由生成器 (G) 和判别器 (D) 组成,两者互相对抗进行优化,生成器将从隐空间采样的变量映射为尽可能接近监督样本的图像,判别器则对生成样本的真实性进行判断。基于 GAN 的方法假设,当训练集中均为正常样本时,模型只能学习到正常样本的分布,因此隐空间变量经过生成器后只能得到正常图像。可根据待测图像与重建图像的误差实现缺陷定位,同时也可根据判别器的判别结果实现缺陷的检出。

AnoGAN\cite{schleglUnsupervisedAnomalyDetection2017}是第一个将GAN引入缺陷检测的方法。其采用迭代优化思路,随机采样一个变量输入到已在正常样本上训练完的DCGAN\cite{radfordUnsupervisedRepresentationLearning2016}中,得到生成图像并计算重建误差。通过反向传播更新隐空间变量,以找到最相近的隐空间变量。若样本经过 n 次迭代后重建误差仍然较大,则判为异常。但此方法由于优化过程需要反复迭代,大幅增加了模型推理时间,缺乏实用性。对此,Eff-GAN\cite{zenatiEfficientGANBasedAnomaly2019} 和 f-AnoGAN\cite{schleglFAnoGANFastUnsupervised2019} 在 GAN 的基础上引入了编码器,提取图像特征,以指导生成器生成最匹配的图像。f-AnoGAN 采用了多阶段的训练方式,第一阶段在正常样本上训练 WGAN \cite{gulrajaniImprovedTrainingWasserstein2017},第二阶段训练编码器以生成与输入最接近的正常图像。Eff-GAN 则将生成器的结构倒置作为编码器,训练时根据重建损失和判别损失共同约束进行端到端的训练。得益于无需繁复的迭代过程,推理速度提升明显。为保证重建图像的一致性,Komoto等\cite{komotoConsistencyEnsuredBidirectional2020}提出了将BI-GAN\cite{donahueAdversarialFeatureLearning2017}的生成器用AE模型代替的方法,实现图像空间和隐空间双向映射,其编码器提取的特征即为图像在隐空间的最佳表达。

% 把AE或VAE与GAN结合框架是当下主流趋势。Baur等\cite{baurDeepAutoencodingModels2019}将VAE与GAN结合,利用重建误差分割缺陷区域。GANomaly\cite{akcayGANomalySemisupervisedAnomaly2019}增加对抗训练并引入额外编码器,提升重建一致性。Skip-GANomaly\cite{akcaySkipGANomalySkipConnected2019}借鉴U-Net结构,结合多尺度特征,提升检测能力。DAGAN\cite{tangAnomalyDetectionNeural2020}参考BEGAN\cite{berthelotBEGANBoundaryEquilibrium2017},将判别器改为AE结构,适用于小工业数据集。Zhou等\cite{zhouEncodingStructureTextureRelation2020}引入结构提取模块,强化结构与纹理关系,提升重建质量和缺陷定位判断。Bergmann等\cite{bergmannImprovingUnsupervisedDefect2019}使用结构相似性损失函数,提升检测复杂真实工业数据能力。

由于模型容量大且缺陷与正常区域的特征差别小,即使仅在正常样本上训练,模型仍然可能将未见过的缺陷完整重建,因此基于图像重建的方法并非完全可靠。目前主要有两种思路来解决这一问题:一种方法是对隐空间分布施加约束,使其只对正常样本具有良好表示。另一种方法是引入特征存储器 (Memory) 来存储正常特征。

改进 Memory 方法利用额外空间存储正常训练样本特征,使模型重建正常图像。推理时,待测图像样本特征与 Memory 中匹配,利用其重建可以保证重建图像不会包含缺陷。Gong 等\cite{gongMemorizingNormalityDetect2019}提出的 MemoryAE比直接存储图像更加灵活,且节约内存。 Memory 的结构与更新机制是后续工作改进的方向。

基于图像重建方法简单直观,无需复杂预处理,但其输入与输出难以对齐,鲁棒性不好,易出现像素级误检。其次,AE 等模型虽基于正常样本训练,仍可能泛化到缺陷,甚至退化成恒等映射而无法检测异常。MemoryAE 方法存储大量正常样本特征,虽然一定程度上解决了恒等映射问题,但也导致模型开销大且效率低。

% 基于图像恢复方法将缺陷视为噪声,图像恢复则是去噪过程。此类方法的核心思想是在正常图像上加入缺陷后,训练网络模型将其恢复为对应的原始图像。常用的模型包括 AE 和 U-Net 等。训练完成后,模型具有根据上下文消除缺陷的能力。测试阶段利用恢复图像和输入图像的重建误差进行缺陷分割。由于模型的输入与输出不对等,此类方法能一定程度上避免恒等映射的问题。但其要求人工缺陷设计接近真实分布且具多样性,避免过拟合。其性能受恢复图像对齐程度影响,在实际工业部署时难满足实时性要求,效率较低。

\subsubsection{基于特征相似度的方法}

基于图像相似度的方法直观且可解释,但现有方法难以实现理想的重建效果,可能出现未对齐或改变风格的情况,导致检测错误并限制性能。利用特征空间比较高维特征以实现更鲁棒的检测是另一个方向。深度神经网络的特征提取能力提供了区分性的特征嵌入,减少无关特征的干扰。CNN提取的特征包含局部感受野的信息,不需要非常严苛的空间对齐,同时增加了对噪声的容忍能力。

% 深度一类分类方法结合神经网络与传统分类方法,提取正常图像特征并构建分界面。传统方法在大数据集或高维数据中检测异常时存在局限,深度学习可以帮助从高维工业图像中提取区分特征。早期混合方法\cite{erfaniHighdimensionalLargescaleAnomaly2016}先用自编码器学习特征,再用OC-SVM构建分界面。Deep SVDD\cite{ruffDeepOneClassClassification2018}结合深度学习与SVDD,训练神经网络将正常样本映射到超球内,但方法的特征中心点需人为指定,可能存在模型退化等。可以通过优化特征提取来规避局限\cite{kitamuraExplainableAnomalyDetection2019}。利用自监督学习设计监督信息与代理任务可以提升分类性能。此类方法用于工业缺陷检测任务时,需要对工业数据进行针对性的设计,主要难点在于挖掘和优化区分性特征。

特征距离度量方法不需要优化分界面,直接与待测样本进行比较。因此,需要合适的特征表达。使用正样本训练的特征提取器无法学习缺陷的特征,导致模型无法区分。因此,研究者选择使用预训练模型来描述图像的特征。度量形式主要有两种:基于向量的距离和基于分布的距离。

一种直观的方法是将待测图像与最相似的正常模板进行比较。该方法使用具有区分力的特征提取网络将多个输入图像映射到特征空间,它们的特征仅在缺陷区域差别明显。Napoletano 等\cite{napoletanoAnomalyDetectionNanofibrous2018}对正常图像划分区域,使用预训练模型提取特征、降维与聚类等操作来构建正常字典,以处理具有多样性的正常图像。Cohen 等\cite{cohenSubImageAnomalyDetection2021}提出的 SPADE 方法使用 kNN 法检索与待测样本语义最相似的 k 个正常样本,最后采用特征金字塔匹配的方法进行多尺度特征对齐,从而提升了缺陷定位的鲁棒性。这种方法简单直接,分割性能良好而稳定,尤其适用于固定机位的工业检测场景。但是,其在大型复杂数据集上的应用受到存储开销和检索时间的限制。

基于深度统计模型的方法通过对正常样本的特征进行概率分布建模来减少开销。Rippel等\cite{rippelModelingDistributionNormal2021}使用预训练网络提取正常样本的多尺度特征,并将各个特征图分别建模为多元高斯分布。测试阶段使用待测样本特征向量与正常分布的马氏距离来度量样本是否存在缺陷。Dedard等\cite{defardPaDiMPatchDistribution2021}将多元高斯分布建模于图像块的粒度上,利用多层特征图对齐组成的超列来融合多尺度特征,使得其存储和检索开销都比SPADE\cite{cohenSubImageAnomalyDetection2021}少。此方法尽管能够检测结构缺陷,但泛化性不佳。Rudolph等\cite{rudolphSameSameDifferNet2021}使用归一化流(NF)\cite{rezendeVariationalInferenceNormalizing2016}模型提升灵活性。该方法使用预训练模型提取正常样本的特征,然后基于NF模块将正常样本的特征映射到潜在空间Z,使其服从高斯分布。该方法假设异常样本的特征在Z空间将不服从高斯分布所以似然较小,从而可以敏锐地检测出微弱的缺陷,且无额外的存储开销。

另一种思路是将同一测试图像映射到特征空间的不同方式。该方法假设将图像映射到正常区域的结果相似,而对缺陷区域的映射结果差别较大。Bergmann等人\cite{bergmannUninformedStudentsStudentTeacher2020}首次将教师-学生框架应用于缺陷检测中。他们将大型预训练网络的表征能力通过知识蒸馏的方式迁移到轻量的教师网络中,并仅在正常数据集上训练多个随机初始化的学生网络,以使它们和教师网络在正常样本上的表达一致。其假设学生网络与教师网络在缺陷样本的表达具有较大的回归误差,且同时多个学生网络之间对缺陷的表达存在较大的不确定性,从而实现像素级的缺陷分割。为了解决感受野固定的问题,Salehi等人\cite{salehiMultiresolutionKnowledgeDistillation2021}直接将大型预训练模型作为教师网络,使用轻量紧凑的学生网络,以提升推理速度并专注于区分性特征。使用模型解释性方法,如SmoothGrad\cite{smilkovSmoothGradRemovingNoise2017}等,使模型可以适应多种尺度的缺陷,并极大地降低了计算复杂度,实现了实时检测。Gudovskiy等人\cite{gudovskiyCFLOWADRealTimeUnsupervised2022}提出了一种基于一个条件归一化流框架的实时无监督缺陷检测模型CFLOW-AD ,其包括一个经过鉴别预训练的编码器,以及一个多尺度生成解码器,后者明确地估计编码特征的可能性。

\subsubsection{3D缺陷检测}

相比于大量关于2D异常检测方法的研究,3D缺陷检测相关研究相对较少,是未来发展的趋势。3D缺陷检测较早的工作是研究者在医学成像研究中将缺陷检测方法应用到体素数据中。Simarro等人\cite{simarrovianaUnsupervised3DBrain2021}将应用于2D图像的f-Anogan\cite{schleglFAnoGANFastUnsupervised2019}扩展到3D。Bengs等人\cite{bengsThreedimensionalDeepLearning2021}提出了一种针对医学体素数据的3D自编码器方法,其空间消除方法可进一步提高缺陷检测性能并减少对大型数据集的要求。

然而体素数据与点云3D数据仍然有较大差异,Bergmann等人\cite{bergmannMVTec3DADDataset2022}认识到缺乏用于3D点云数据异常分割的数据集,推出了MVTec 3D-AD,为3D缺陷检测算法的研究打开了新局面。在该数据集上,Bergmann和Sattlegger \cite{bergmannAnomalyDetection3D2023}将教师学生网络扩展到3D点云缺陷检测,称为$3D-ST_{128}$。Horwitz等人\cite{horwitzBackFeatureClassical2022}将2D图像的PatchCore方法扩展到3D,其使用FPFH作为3D特征提取和预训练的网络作为RGB图像特征提取。此外,其做了多项对比实验证明加入3D数据确实有利于提升缺陷检测性能且使用简单的3D特征提取描述子即可取得不错效果。Rudolph等人\cite{rudolphAsymmetricStudentTeacherNetworks2022}提出异构的教师学生网络AST,教师网络使用标准化流而学生网络使用简单残差网络,其假设教师模型在正常样本上训练,学生通过回归教师网络模型,两者在缺陷样本上将会有较大差异,从而实现缺陷检测。AST网络利用EfficientNet提取二维图像特征作为输入,深度图像只混合原始深度信息以达到和二维特征图匹配。实验表明,AST在多模态缺陷检测上性能优异。

% \subsubsection{自监督}

% 自监督学习属于无监督学习的一种。其利用无标注数据自动设计监督标签,并通过类似有监督学习的方式训练模型。主要可以分为三种方法:基于图像复原、基于缺陷合成和基于图像变换的方法。

% 基于图像复原的方法可以挖掘图像内在属性与内容的关联性。其主要思想是显式地训练模型,将经过变换的输入图像复原为原始正常图像。在测试阶段,正常区域的像素基本不变,而异常区域被擦除,进而根据重建误差实现缺陷的检测与分割。这类方法展现出更强的泛化性,但在测试时需要遍历图像的预处理过程,影响了模型的效率。

% 基于缺陷合成的方法可以在正常图像上构造缺陷,得到缺陷样本及其图像或像素级精确的标签。然后可以利用对比学习或有监督的方式训练模型。测试时,分别在图像与图像块维度进行高斯密度估计,以实现缺陷的检测与定位。然而,缺陷合成的方法对于合成方式非常敏感,往往容易过拟合到人造缺陷模式上。

% 基于图像变换的方法,通过代理任务训练判别式模型。该方法假设模型能够正确判别正常样本的变换方式,而无法预测异常样本的变换方式。这类方法往往会与深度一类分类方法结合。常见的变换方式包括旋转角度、颜色变化、顺序打乱等,将变换的编号或参数作为监督信息,训练网络预测正常图像的变换方式。然而,由于图像变换往往是全局的,此类方法通常用于图像级的缺陷检测。


\subsection{特征提取的研究}
针对RGB图像的特征提取方法主要分为传统方法和基于学习的方法。\cite{chenSurfaceDefectDetection2021}传统特征提取方法主要有基于纹理特征的方法、基于颜色特征的方法和基于形状特征的方法三类方法。

纹理特征反映了图像的同质性现象,基于纹理特征的方法包括通过直方图、灰度共生矩阵等特征描述灰度值的空间分布的统计方法、利用傅里叶变换和小波变换等的信号处理方法和对图像建模来描述纹理特征的模型方法。颜色特征计算量小,不受图像大小、方向、视角等因素影响,具有高鲁棒性。基于颜色特征的方法有表示图像中不同颜色比例的颜色直方图、使用矩阵表示颜色分布的颜色矩和改进颜色直方图算法的颜色相干向量。基于形状的方法可以用于目标检索,其中轮廓法通过霍夫变换和傅里叶形状描述子来获得图像的形状参数,前者利用图像的全局特征将边缘像素连接起来形成闭合边界,后者则利用对象边界的傅里叶变换作为形状描述,将二维问题转化为一维问题。

Lowe等人\cite{loweDistinctiveImageFeatures2004}提出了具有旋转、尺度和平移不变性的SIFT,其在图像中识别关键点或兴趣点,通过旋转将最显著的方向与基本方向对齐来减少旋转歧义,从而实现在旋转图像之间进行匹配。

Dalal等人\cite{dalalHistogramsOrientedGradients2005}提出了一种特征描述子HOG(Histograms of Oriented Gradients),其与SIFT等局部图像描述符类似,但是计算方式不同,可以更有效地进行密集采样。HOG通过将图像分成称为单元格的小连通区域,并计算每个单元格内像素的梯度方向或边缘方向的直方图。 

基于学习的方法通常使用在大型数据集上预训练的神经网络模型来提取图像的特征图。He等人\cite{heDeepResidualLearning2016}提出了可用于图像分类任务的ResNet,其过跳过连接将一个层的输出添加到不一定相邻的另一个层中来学习残差函数,解决了梯度消失问题,是网络可以变得深层。

Tan等人\cite{tanEfficientNetRethinkingModel}提出了基于一种简单但高效的复合缩放方法的EfficientNet,并通过神经架构搜索设计了一个新的基线网络,将其扩展为一组模型,称为 EfficientNets。该方法使用一组固定缩放系数统一缩放深度、宽度、分辨率的所有维度,实验表明该方法在 MobileNets 和 ResNet 扩展上的有效性。

针对三维数据的特征提取方法主要有基于空间特征的方法和基于几何特征的方法。\cite{georgiouSurveyTraditionalDeep2020,guoNovelLocalSurface2015,guoComprehensivePerformanceEvaluation2016,hutchisonUniqueSignaturesHistograms2010,liu3DImagingAnalysis2020,rusuAligningPointCloud2008,rusuFastPointFeature2009,tombariCombinedTextureshapeDescriptor2011,yangEffectSpatialInformation2017,yangTOLDIEffectiveRobust2017,yulanguoRoPSLocalFeature2013,zhangNotAllPoints2022,zhaoHoPPFNovelLocal2020}基于空间特征的方法通过根据表面上的点的空间分布(例如坐标)生成直方图来表示局部表面。这种方法通常从为关键点构建局部参考框架(LRF)开始,并根据 LRF 将 3D 支持区域划分为多个区间。在每个空间区间中累积空间分布测量(例如点数)以生成局部表面的直方图。\cite{guoComprehensivePerformanceEvaluation2016}

Guo等人\cite{yulanguoRoPSLocalFeature2013}提出了一种基于旋转投影统计的3D特征描述符RoPS(Rotational Projection Statistics),其通过将相邻的3D点旋转投影到2D平面上,并计算这些平面上分布矩阵的低阶矩和熵来构建的,对噪声和网格分辨率的变化具有鲁棒性。

Guo等人\cite{guoNovelLocalSurface2015}还提出了一种能够在杂波和遮挡的情况下识别3D物体的局部特征描述子TriSI(Tri-Spin-Image),其通过生成三个签名来编码局部表面的几何信息,然后这些签名被连接起来形成原始的TriSI特征向量,并使用主成分分析(PCA)技术进一步压缩该特征。

基于基于几何特征的描述子通过根据表面上点的几何属性(例如法线,曲率)生成直方图来表示局部表面。

Rusu等人\cite{rusuAligningPointCloud2008}研究了使用PFH(Persistent Feature Histograms)来将点云数据视图对齐到一致的全局模型。给定一组嘈杂的点云,通过分析不同尺度下特征的持久性,估计一组描述每个点局部几何的稳健的多维特征。

Rusu等人\cite{rusuFastPointFeature2009}在点云配准的背景下,又提出了FPFH(Fast Point Feature Histograms),该方法通过计算其邻居的相对方向直方图来描述点云中一个点的局部几何特性。FPFH是PFH描述子的扩展,更具计算效率和噪声鲁棒性。





\subsection{联邦学习的研究}
随着近年来工业界对数据隐私及安全问题越来越重视,数据孤岛现象越来越多,而这与深度学习对大量训练数据的需求矛盾,制约了深度学习在工业缺陷检测的落地。联邦学习作为能够保护隐私、解决数据孤岛问题的一种分布式机器学习方法成为研究热点。\cite{banabilahFederatedLearningReview2022,liFederatedLearningChallenges2020,liuRecentAdvancesFederated2023,mammenFederatedLearningOpportunities2021}

联邦学习只在本地利用数据样本进行训练,不上传共享数据而是通过上传共享模型参数实现在多个分散设备或服务器上分布式学习,从而解决在大量分散数据上进行训练时面临的隐私和安全问题,确保数据隐私和安全性。

谷歌提出了联邦学习的概念,介绍了一种实用方法FedAvg\cite{mcmahanCommunicationEfficientLearningDeep2023},该方法允许用户共享经过训练的模型,而无需将数据集中存储。FedAvg将训练数据分布在移动设备上,并通过聚合本地计算的更新来学习共享模型。实验表明,FedAvg对于不平衡和非独立同分布数据具有鲁棒性。在五种不同的模型架构和四个数据集上进行了广泛的实证评估。

联邦学习可分为同构和异构两种类型,根据输入数据集、本地网络模型和设备类型。数据异构性可能导致本地节点处于不同情况下,从而导致各种数据分布。多任务学习可同时为多个相关任务学习模型,其核心设计原则是捕获任务之间的关系并促进学习过程。客户端的不同数据分布可以被视为一种多任务学习。

在FedAvg之后,联邦学习的研究主要有两个方向:解决客户端数据异构性的问题,设计联邦学习优化算法以加快模型收敛速度;解决联邦学习模型上传和下载中的通信成本高问题,设计提高通信效率的算法。此外,为了方便研究实验,还提出了一些联邦学习相关框架。\cite{zengFedLabFlexibleFederated2022}

目前联邦学习聚合优化主要有权重级聚合、特征级聚合和其它聚合\cite{sattlerRobustCommunicationEfficientFederated2019,wangFederatedLearningMatched2020,zhaoFederatedLearningNonIID2022}。FedAvg是典型的权重级聚合方法,Collins等人\cite{collinsFedAvgFineTuning2022}研究了FedAvg算法的泛化能力,其能够利用客户端数据分布的多样性学习客户端任务之间的共同数据表示,从而输出具有表示学习能力的模型。还证明了FedAvg在共享表示是线性映射的情况下仍能学习到泛化到新客户端的共享表示以及其在异构数据的联邦图像分类中的表示学习能力。

Wu等人\cite{banabilahFederatedLearningReview2022}介绍了一种名为FedKD的通信高效联邦学习方法。FedKD通过知识蒸馏有效地降低了通信成本。在FedKD中,客户端和服务器不直接传递大模型,而是提取了一个小学生模型和一个大教师模型。只有学生模型被共享和协同学习。每个客户端都维护一个本地教师模型和共享学生模型的本地副本,通过自适应知识蒸馏方法从本地数据和彼此提取的知识中学习。此过程将被迭代执行,直到学生模型收敛。为了降低学生模型更新时的通信成本,还提出了一种基于奇异值分解(SVD)的动态梯度近似方法来压缩通信梯度。

Han等人\cite{hanFedXUnsupervisedFederated2022}提出了一种无监督的联邦学习算法FedX,通过本地和全局两个层面上的独特的两面式知识蒸馏来学习数据表示。两面式知识蒸馏可以从本地数据中发现有意义的表示,同时通过使用全局知识消除偏见。FedX可以应用于现有算法,实验表明模型在五种无监督算法上显著提高了性能,并进一步提高训练速度。

在联邦学习通信与压缩方面\cite{linDefensiveQuantizationWhen2019,linDeepGradientCompression2020,reisizadehFedPAQCommunicationEfficientFederated2020,zhouCommunicationefficientFederatedLearning2023,zhouCommunicationefficientFederatedLearning2023},Haddadpour等人\cite{haddadpourFederatedLearningCompression2021}提出了采用周期性压缩(量化或稀疏化)通信的方法FedCOM,并分析其在同质和异质本地数据分布设置下的收敛性质。作者阐述了梯度压缩和周期性通信中本地计算的关系。FedCOM通过周期性平均、本地和全局学习速率以及压缩通信,在同质设置中比现有方法收敛更快。

在联邦学习框架方面,Zeng等人\cite{zengFedLabFlexibleFederated2022}等人提出了一种轻量级的开源联邦学习框架FedLab。该框架的设计侧重于FL算法的效率和通信效率,并且在不同的部署场景中具有可扩展性。该框架提供了FL模拟所需的必要模块,包括通信、压缩、模型优化、数据分区和其他功能模块。FedLab提供标准同步和异步FL系统的标准化FL实现方案以及FL数据集基准和标准FL模拟的功能模块,减轻了联邦学习社区中研究人员实现新方法的负担。


\section{论文研究内容与组织结构}
\subsection{论文研究内容}
本文面向工业缺陷检测中非平整表面的复杂产品,针对其缺陷样本匮乏、结构化缺陷检测性能有待提升的两个痛点,搭建视觉实验平台,采集点云数据和RGB-D多模态数据,分别研究仅利用正常样本数据实现工业缺陷检测的算法,并探索了将联邦学习应用到本文算法实现分布式缺陷检测。本文主要研究内容如下:

(1)搭建缺陷检测实验平台

实验平台是本文缺陷检测算法研究的基础。本文研究通过协作机器人和结构光激光相机组合搭建实验硬件平台,并设计了带有交互界面的自动数据采集应用,提供后需研究方法所需的标准数据集。除了数据采集平台,本文还搭建了缺陷算法运行测试的软件平台。

(2)研究基于3D点云数据的缺陷检测算法

非平整表面的工业产品结构非常复杂,其缺陷也主要以结构缺陷为主,二维图像适用的缺陷检测方法对此性能不佳,因此需要一种更加高效的方法来表征这种结构。本文旨在探讨如何利用点云数据来表征工业产品的结构。首先,本文对点云数据进行了合理的预处理,以提高点云的计算效率。接着,本文引入了一种基于空间的点云特征描述子RoPS,该描述子可以提取点云的特征。最后,本文使用Patchcore的骨干对提取的特征进行存储,从而实现对未知缺陷的检测。

(3)研究基于RGB-D数据的多模态缺陷检测算法

除了结构缺陷,有时非平整表面的工业产品也会有颜色和纹理等非结构的缺陷,单纯使用2D或3D数据都不足以满足需求。因此,本文采用一种多模态融合的方法来解决问题。首先,本文采集了RGB-D的多模态数据,然后分别利用预训练的模型提取RGB图像的特征和引入HOG提取深度图像的特征,从而可以更加全面地表征工业产品多种特征。接着,本文将这些特征与位置编码拼接,送入异构的教师学生网络训练模型来实现多模态的缺陷检测,其中学生网络通过新增深度特征卷积层来改进多模态融合。

(4)研究基于联邦学习的分布式缺陷检测算法

分布式缺陷检测检测是工业缺陷检测未来的发展趋势。本文研究将联邦学习应用到本文改进的异构教师学生缺陷检测算法上来实现分布式缺陷检测,并通过联邦学习仿真框架利用大型数据集模拟多个分布式产线,从而验证本文的有效性和探索其扩展性。

\subsection{论文组织结构}
\begin{figure}[htbp]
    \centering
    \includegraphics[width=0.75\textwidth]{figures/1/paper-framework.pdf}
    \caption{论文组织结构图}\label{fig:paper-framework}
  \end{figure}
本文面向非平整表面的缺陷检测算法研究从背景出发搭建实验平台,研究不同场景下缺陷检测算法并应用分布式方法部署算法,整个论文的组织结构如图\ref{fig:paper-framework}所示。

第一章绪论介绍了面向非平整表面缺陷检测的研究背景和意义,并对国内外未知缺陷检测算法、特征提取和联邦学习研究现状和发展趋势进行了总结,从而提出了基于3D点云数据和多模态数据两种场景下未知缺陷检测算法的研究以及应用分布式联邦学习的思路。最后概括介绍了本文的研究内容和组织结构。

第二章首先从搭建视觉缺陷检测系统的核心传感器结构光相机的原理出发,介绍了本文实验平台的硬件平台和软件平台并进行了相机标定。此外实验平台主要是采集数据集为本文算法验证提供支持,因此介绍了本文开发的具有交互界面的数据自动采集应用。最后介绍了用于实验中衡量缺陷检测算法性能的评价指标。

第三章首先介绍了基于3D点云数据的缺陷检测算法框架,然后介绍了对输入算法的点云数据进行预处理的滤波算法和将点云贪婪三角化的原理,接着介绍了本文引入的RoPS局部特征描述子的原理和缺陷检测骨干模型的原理。最后在第二章搭建的实验平台上开展了对比实验,证明了本文所提算法的有效性。

第四章研究了基于多模态的缺陷检测,同样从算法整体框架开始,然后介绍对输入多模态数据的预处理,利用预训练网络对RGB图像的特征提取原理和用HOG对深度图像的特征提取的原理,接着分别介绍了使用标准化流的教师网络和本文改进的深度特征融合学生网络。最后也在本文的实验平台上开展了对比实验,验证了本文改进算法对缺陷检测性能的提升。

第五章从联邦学习应用到异构教师学生网络的框架开始,介绍了联邦学习范式的理论基础,然后说明了本文使用的联邦学习全局聚合优化方法和通讯压缩方法的原理,接着介绍了联邦学习的仿真框架,并最后在该框架上进行了实验,证明了利用联邦学习实现缺陷检测算法分布式部署的可行性。

最后结论部分对本文的研究内容和创新点进行了总结和分析,并指出了本文研究中有待改进之处以及本文尚未深入探索的研究方向。